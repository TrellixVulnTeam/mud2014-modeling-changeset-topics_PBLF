\documentclass[conference]{IEEEtran}

\usepackage{balance}
\usepackage{booktabs}
\usepackage{cite}
\usepackage{color}
\usepackage{comment}
\usepackage[pdftex]{graphicx}
\usepackage{url}
\usepackage{listings}


\definecolor{lightred}{RGB}{150,0,0}
\definecolor{lightgreen}{RGB}{0,150,0}
\definecolor{lightblue}{RGB}{0,0,150}

\lstdefinelanguage{diff}{
  morecomment=[f][\color{lightblue}]{@@},     % group identifier
  morecomment=[f][\color{lightred}]-,         % deleted lines
  morecomment=[f][\color{lightgreen}]+,       % added lines
  morecomment=[f][\color{black}]{---}, % Diff header lines (must appear after +,-)
  morecomment=[f][\color{black}]{+++},
}
\hyphenation{}

\newcommand{\attn}[1]{{\color{red}#1}}
\newcommand{\desc}[1]{{\emph{\color{blue}#1}}}
\newcommand{\needcite}{\attn{\tiny{[cite]}}}


\begin{document}
\title{Topic Modeling Changesets}
\author{
    \IEEEauthorblockN{
        Daniel S. May
    }
    \IEEEauthorblockA{
        Department of Computer Science\\
        Swarthmore College\\
        Swarthmore, PA 19081\\
        \texttt{dmay1@swarthmore.edu}
    }

    \and

    \IEEEauthorblockN{
        Kelly L. Kashuda,
        Christopher S. Corley
    }
    \IEEEauthorblockA{
        Department of Computer Science\\
        The University of Alabama\\
        Tuscaloosa, AL 35487\\
        \texttt{\{klkashuda,cscorley\}@ua.edu}
    }

    \and

    \IEEEauthorblockN{
        Nicholas A. Kraft
    }
    \IEEEauthorblockA{
        ABB Corporate Research\\
        Software Research Area\\
        Raleigh, NC 27606\\
        \texttt{nicholas.a.kraft@ua.abb.com}
    }
}


\maketitle

\begin{abstract}

Topic modeling has been applied to several areas of software engineering,
such as bug localization, feature location, triaging change requests,
and traceability link recovery.
Many of these approaches combine mining unstructured data, such as bug
reports, with topic modeling a release snapshot of source code.
However, source code evolves, causing the models to quickly go stale.
In this paper, we explore the approach of topic modeling \emph{changesets}
over the "traditional" release approach.
We conduct an exploratory study of four open source systems:
three written in Java and the last in C.
We investigate the differences in corpora in each project,
and evaluate the topic distinctiveness of the models.
We find that we are all really cool people.

\end{abstract}

\begin{IEEEkeywords}
\end{IEEEkeywords}

\section{Introduction}
\label{sec:intro}
% vim:syntax=tex
Software developers commonly find themselves dealing with large repositories that date back extensively, before delving into a project or specific task (e.g. feature location, maintenance, etc.).
Navigating through these repositories can be a time consuming task since their organization, or lack there of, can be hard to understand.
Topic models have been used to address this problem by identifying the underlying topics in the repositories.
These underlying topics are found using a statistic analysis, through the use of algorithms that are designed to uncover the thematic structure that exist.
Finding this thematic structure helps developers to navigate and better understand the repositories with which they are working.
Topics are discovered by taking a distribution over the vocabulary, which creates a set of topics for each document in the corpus.
With each document representing multiple topics, connections are made between the topics and subtopics, making the documents easier to navigate.
While topic models have addressed these problems (e.g., feature location program comprehension, maintenance, etc.) models are typically built on a snapshot of code, or a release.
A problem with this method of generating topic models is the drastic amount of changes that occur in between releases due to the dynamic nature of source code, which force the models to be rebuilt from scratch.
Given the nature of these problems, we propose that the individual changesets should be used to continually update the source code.
A changeset can be thought of as indexing the documents where changes have occurred, and transforming the indexes from version A to version B.
By doing this, we gain a topic model that evolves dynamically over time without requiring the reprocessing of the entire corpus.

The goal of this paper to determine if a topic model based on the changesets between commits can produce a topic model that is of equal quality to the common topic modeling based on analyzing releases.
In this paper, we will compare results of a standard release model and our changeset model.
The metric that we will use to determine quality is topic distinctness,
which has been developed and used in topic modeling of source
code~\cite{Wei-etal:2010, Thomas-etal:2011, Chuang-etal:2012}.
Topic distinctness measures how distinct a newly discovered topic is in
comparison to the other, already discovered topics.
We use this metric as a  tool to compare our changeset model to a model built upon a release.



% let's not elude to this yet
%Our ultimate goal is to develop a topic model that will allow authors to
%be linked to certain topics and features in the source code (info about
%the diagram Chris always has to draw on white board).



\section{Background \& Related Work}
\label{sec:related}
% vim:syntax=tex

In this section we provide an overview of latent Dirichlet allocation and review closely related work.

\subsection{Latent Dirichlet Allocation}

Latent Dirichlet allocation (LDA)~\cite{Blei-etal:2003} is a generative topic model.
LDA models each document in a corpus of discrete data as a finite mixture over a set of topics
and models each topic as an infinite mixture over a set of topic probabilities.
That is, LDA models each document as a probability distribution
indicating the likelihood that it expresses each topic and
models each topic that it infers as a probability distribution
indicating the likelihood of a word from the corpus being assigned to the topic.

%Inputs to LDA include a corpus and $K$, the number of topics. LDA represents each document in the corpus as a bag-of-word (multiset) and thus disregards word order and structure. Outputs of LDA include $\phi$, the term-topic probability distribution, and $\theta$, the topic-document probability distribution.


\subsection{Topic Models in Software Maintenance}

%Feature location as presented by Rajlich et al.\ is a way of locating concepts within code to increase understanding of the program as a whole~\cite{Rajlich-Wilde:2002}.
%Linstead et al.\ outlined a statistical model using LDA that was able to mine these concepts from source code~\cite{Linstead-etal:2007b}.
%Linstead et al.~\cite{Linstead-etal:2007} used author-topic models to retrieve
%developer contribtions from source code.
%Lukins et al.~\cite{Lukins-etal:2008} implemented a way of using LDA to locate bugs in source
%code that performed better than LSI-based information-retrieval
%techniques.
%Basset et al.~\cite{Bassett-Kraft:2013} extended this work
%and studied various weightings of various terms in source code
%to improve LDA-based feature location accuracy in five Java systems.

Thomas et al.~\cite{Thomas-etal:2011} describe an approach to modeling the evolution of source code topics using LDA. Their \textit{Diff} model outperforms the Hall topic evolution model~\cite{Hall_etal:2008} in the context of software repositories, because the \textit{Diff} model trains topic models on the changesets between two snapshots, rather than on individual snapshots. That is, for a particular source code file, \textit{Diff} trains a topic model on a document that represent the changes between consecutive versions of the file. Consequently, the \textit{Diff} model eliminates the issues related to data duplication that arise in the Hall model, which trains a topic model on all versions of the (whole) file. Thomas et al.\ demonstrate the efficacy of the \textit{Diff} model via a comparative study with the Hall model. Their evaluation measures include topic distinctness, which we define in Section~\ref{sec:study}.

Hindle et al.~\cite{Hindle_etal:2012} validate the use LDA topics during software maintenance via a study at Microsoft. Their focus is on stakeholder validation of topics --- i.e., they seek confirmation that LDA topics are interpretable by stakeholders (e.g., developers or managers) and relevant to the requirements implemented by the modeled source code. Previous work by Hindle et al.~\cite{Hindle-etal:2009} describes an approach to modeling the evolution of software topics using commit messages rather than source code.

\begin{table*}[ht]
\renewcommand{\arraystretch}{1.3}
\footnotesize
\centering
\caption{Subject systems version and corpora description}
\begin{tabular}{lll rr rr rr}
    \toprule
            & Snapshot & Commit & Snapshot No.  & Changeset No. & Snapshot No.    & Changeset No.  & Snapshot No.    & Changeset No.\\
     System    & Version & SHA  & Documents   & Documents    & Unique Terms  & Unique Terms  & Total Terms   & Total Terms \\
    \midrule
    \ant        & 1.9.4   & 1c927b15 & 2208      & 12,996     & 17,986         & 74,681         & 1,066,446       & 11,801,353 \\
    \aspectj    & 1.8.0   & 5a5bef1e & 10130     & 7,650      & 22,855         & 25,071         & 4,825,289       & 10,583,008 \\
    \jodatime   & 2.3     & b0fcbb95 & 402       & 1,750      & 9,298          & 11,385         & 493,131        & 5,541,330 \\
    \postgres  & 9.3.4   & d4f8dde3 & 4080      & 36,870     & 84,591         & 164,703        & 6,644,409       & 59,850,328 \\
    \bottomrule
\end{tabular}
\label{tab:systems}
\end{table*}

Although our work is preliminary, we believe that it is the first to consider modeling changesets in lieu of snapshots to support software maintenance. Like Rao et al.~\cite{Rao-etal:2011}, we are targeting problems that require an up-to-date topic model. Thus, the expense of training a topic model is a key consideration for us, unlike for Thomas et al.~\cite{Thomas-etal:2011} or Hindle et al.~\cite{Hindle-etal:2009,Hindle_etal:2012}.


\section{Case Study}
\label{sec:study}
% vim:syntax=tex

In this section we describe the design of a case study in which we
compare topic models trained on changesets to those trained on snapshots.
%explore the relationship between ownership and linguistic topics in source code.
We describe the case study using the Goal-Question-Metric approach~\cite{Basili-etal:94}.
%The data for the case study is available in this paper's online
%appendix\footnote{\url{http://software.eng.ua.edu/data/ownership-topics}}.

\subsection{Definition and Context}

Our \textit{goal} is to explore the relationship between changeset topics and snapshot topics.
The \textit{quality focus} of the study is on informing development decisions and policy changes
that could lead to software with fewer defects.
The \textit{perspective} of the study is of a researcher, developer, or project manager who wishes
to gain understanding of the concepts or features implemented in the source code.
The \textit{context} of the study spans the version histories of four open source systems.

Toward achievement of our goal, we pose the following research questions:
\begin{description}[font=\itshape\mdseries,leftmargin=10mm,style=sameline]
    \item[RQ1] Do changeset- and snapshot-based corpora express the same terms?
    \item[RQ2] Are topic models trained on changesets more distinct than topic models trained on a snapshot?
\end{description}
At a high level, we want to determine whether topic modeling changesets can perform as well as, or better than, topic modeling a snapshot.

In the remainder of this section we introduce the subjects of our study,
describe the setting of our study, and report our data collection and analysis procedures.

%%%%%%%%%%%%%%%%%%%%%%%%%%%%%%%%%%%%%%%%%%%%%%%%%%%%%%%%%%%%%%%%%%%%%%%%

\subsection{Subject software systems}

The four subjects of our study ---
Apache \ant\footnote{\url{http://ant.apache.org/}},
\aspectj\footnote{\url{http://eclipse.org/aspectj/}},
\jodatime\footnote{\url{http://www.joda.org/joda-time/}},
and \postgres\footnote{\url{http://www.postgresql.org/}}
--- vary in language, size and application domain.
\ant\ is a library and command-line tool for managing builds,
\aspectj\ is an aspect-oriented programming extension for the Java language,
\jodatime\ is a library for replacing the Java standard library \texttt{date} and \texttt{time} classes,
and \postgres\ is an object-relational database management system.
Each system is stored in a Git repository, and the developers of each system use descriptive commit messages.
Further, the developers store bug reports in an issue tracker.
All systems are written in Java with the exception of \postgres,
which is written in C. Table~\ref{tab:systems} outlines the releases for
each system studied.

%%%%%%%%%%%%%%%%%%%%%%%%%%%%%%%%%%%%%%%%%%%%%%%%%%%%%%%%%%%%%%%%%%%%%%%%

\subsection{Setting}

\begin{figure*}[!th]
    \centering
    \includegraphics[width=.75\textwidth]{changeset}
    \caption{Extraction and Modeling Process}
    \label{fig:process}
\end{figure*}

Our document extraction process is shown on the left side of Figure~\ref{fig:process}.
We implemented our document extractor in Python v2.7
using the Dulwich library\footnote{\url{http://www.samba.org/~jelmer/dulwich/}} %\footnote{\url{https://pypi.python.org/pypi/dulwich}}
We extract documents from both a snapshot of the repository at a tagged
release and each commit reachable from that tag's commit.
The same preprocessing steps are employed on all documents extracted.

For our document extraction from a release,
we simply use the entire contents of the document.
Unlike previous methods, %\needcite,
we do not parse the source code documents for classes, methods, and so on.
%We do this because so-and-so-I-forget\needcite,
%but also it leaves our technique language-independent.
We do this to leave our technique language-independent,
and to also allow for a fair comparison between the two approaches.
Special characters such as braces and semicolons are removed during
preprocessing.

To extract text from the changesets, we look at the output of viewing
the \texttt{git diff} between two commits.
Figure~\ref{fig:diff} shows an example of what a changeset might look
like in Git.
In our changeset text extractor, we only extract text from removed or added lines.
Context and metadata lines are ignored.
%Figure~\ref{fig:diffdocument} shows what words would be extracted to make up the document without preprocessing
%from the change shown in Figure~\ref{fig:diff}.
Note that we do not consider what type of document the text originates from,
only that it is text changed by the commit.

%Our document preprocessor implements the transformations described in Section~\ref{sec:preprocessing}.
%We filter \texttt{java.lang} class names before splitting tokens.
After extracting tokens, we split them based on camel case, underscores, and non-letters.
We normalize to lower case before filtering English stop words~\cite{StopWords}, Java keywords, and words shorter than three characters long.
We do not stem.


For our model generation, we use the open source Python library Gensim~\cite{Gensim}.
Gensim's LDA implementation is based on the Online LDA by Hoffman et al.~\cite{Hoffman-etal:2010}
and uses variational inference instead of a Collapsed Gibbs Sampler.
Unlike Gibbs sampling, in order to ensure that the model converges for each document,
we allow LDA to see each document $10$ times by setting Gensim's initialization parameter \texttt{passes} to this value.
We set the following LDA parameters for all four systems:
$100$ topics ($K$),
a symmetric $\alpha=0.01$,
$\beta$ is left as a default value of $1/K$ (also $0.01$).


\begin{figure}[ht]
\centering
\footnotesize
\begin{lstlisting}[language=diff, basicstyle=\ttfamily]
diff --git a/lao b/tzu
index 635ef2c..5af88a8 100644
--- a/lao
+++ b/tzu
@@ -1,7 +1,6 @@
-The Way that can be told of is not the eternal Way;
-The name that can be named is not the eternal name.
 The Nameless is the origin of Heaven and Earth;
-The Named is the mother of all things.
+The named is the mother of all things.
+
 Therefore let there always be non-being,
   so we may see their subtlety,
 And let there always be being,
@@ -9,3 +8,6 @@ And let there always be being,
 The two are the same,
 But after they are produced,
   they have different names.
+They both may be called deep and profound.
+Deeper and more profound,
+The door of all subtleties!
\end{lstlisting}
\caption{Example of a \texttt{git diff}. Black or blue lines denote metadata about the change useful for patching, red lines (beginning with a single~\texttt{-}) denote line removals, and green lines (beginning with a single~\texttt{+}) denote line additions.}
\label{fig:diff}
\end{figure}

\begin{comment}
\begin{figure}[ht]
\em
\footnotesize
The Way that can be told of is not the eternal Way;
The name that can be named is not the eternal name.
The Named is the mother of all things.
The named is the mother of all things.
They both may be called deep and profound.
Deeper and more profound,
The door of all subtleties!
\caption{An example extracted changeset document before preprocessing}
\label{fig:diffdocument}
\end{figure}
\end{comment}

%%%%%%%%%%%%%%%%%%%%%%%%%%%%%%%%%%%%%%%%%%%%%%%%%%%%%%%%%%%%%%%%%%%%%%%%

\subsection{Data Collection and Analysis}

We create two corpora for each of our four subject systems.
We then used LDA to model the documents into topics.

To answer RQ1, we investigate the term frequency in each corpus.
We create two distributions from all unique terms from both corpora.
That is, each vector is of the same length and contain zero values for terms not in its respective corpus.
We measure the similarity of the two word-vectors using cosine distance.

To answer RQ2, we follow Thomas et al.~\cite{Thomas-etal:2011} 
and use topic distinctness to evaluate our topic models.
Distinct topics are topics with dissimilar word probabilities to all other topics in the model.
Using this metric will also allow a comparison to the results of Thomas et al.~\cite{Thomas-etal:2011}. 
Topic distinctness has also been shown to be an effective measure for qualitative
comparison of topic models in context of visualization~\cite{Wei-etal:2010, Chuang-etal:2012}.
Although Chang et al.~\cite{Chang-etal:2009} conclude that evaluating topic models
should depend on real-world tasks over metrics,
we argue that the positive results from using topic distinctness for visualization
techniques qualifies the metric for this exploratory study.

\begin{comment}
``Traditional metrics are, indeed, negatively correlated with the
measures of topic quality developed in this paper.  Our measures enable
new forms of model selection and suggest that practitioners developing
topic models should thus focus on evaluations that depend on real-world
task performance rather than optimizing likelihood-based measures.''~\cite{Chang-etal:2009}
\end{comment}

\newpage
Thomas et al.\ define topic distinctness (TD) of a topic $z_i$ as the mean
Kullback-Leibler (KL) divergence between the vectors $z_i$ and $z_j$, $\forall j \neq i$:
\begin{equation}
TD(\phi_{z_i}) =
\frac{1}{K - 1}
\sum_{j=1,j \neq i}^{K}
KL(\phi_{z_i}, \phi_{z_j})
\label{eq:topicdistinctness}
\end{equation}

We score the overall topic distinctness of a model $\phi$ as the mean of
its topic distinctness scores, $\forall z \in \phi$.

\begin{comment}
\begin{equation}
TD(\phi) =
\frac{1}{K - 1}
\sum_{i=1}^{K}
TD(\phi_{z_i})
\label{eq:overalltopicdistinctness}
\end{equation}
\end{comment}


%%%%%%%%%%%%%%%%%%%%%%%%%%%%%%%%%%%%%%%%%%%%%%%%%%%%%%%%%%%%%%%%%%%%%%%%

\subsection{Results}


RQ1 asks whether a corpus generated from changesets have similar terms
as a corpus generated from a release.
We expected to find that set of unique terms in the changeset corpus to be
a superset of the set of unique terms in the release corpus.
Interestingly, this holds true for \jodatime\ and \aspectj,
but not for \ant\ and \postgres.
Further inspection shows that 2 terms appear in the \ant\ release corpus that do not appear in the changeset corpus,
and 19 terms appear in \postgres's respective corpora.
However, these anomalies appear to have been file encoding errors that
had been introduced into the version history.
For example, one of the terms for \ant\ was ``rapha''. This is due
to an encoding error in the \texttt{KEYS} file for a developer named
``Rapha\"{e}l Luta''. 
Similar encoding errors were found for the remaining 20 terms.

To answer RQ1, we created two word distributions that represented the unique
terms from both corpora for each system.
Figure~\ref{fig:antdist1} shows the normalized distribution of the \ant\ release corpus.
Likewise, Figure~\ref{fig:antdist2} shows the distribution of the \ant\ changeset corpus.
We measure the differences between the two distributions using the simple
cosine distance. For \ant, we had the lowest cosine distance of $0.00396$.
\aspectj\ and \jodatime\ have similar distances to another of $0.06929$ and
$0.06540$. \postgres\ had the largest, $0.33957$.

%
% At this point, I would love to do a stats test. But I am way too tired
% to figure that out because I am butts at stats
%

\begin{figure*}[ht]
\centering
\begin{subfigure}[b]{0.4\textwidth}
    \includegraphics[width=\textwidth]{dist1}
    \caption{Normalized distribution of the release corpus}
    \label{fig:antdist1}
\end{subfigure}
\begin{subfigure}[b]{0.4\textwidth}
    \includegraphics[width=\textwidth]{dist2}
    \caption{Normalized distribution of the changeset corpus}
    \label{fig:antdist2}
\end{subfigure}
\caption{Comparison of \ant\ word distributions}
\vspace*{-12pt}
\end{figure*}


RQ2 asks if topic models trained on changeset corpora produce more distinct
topics.
We expected to find that they would, due to ``popular'' words becoming
more prevalent in the corpus after appearing in several changes.
Interestingly, this is true for \ant\ and \postgres,
but not for \jodatime\ and \aspectj.
We hypothesize this is because \ant\ and \postgres\ have drastically more documents in their respective change set corpora than \jodatime\ and \aspectj.
That is, we think that the feasibility of using changeset topics is somewhat
dependent on the amount of history in the repository.
%Table~\ref{tab:systems} shows the document count and term count for each system.

We can compare directly to the results of Thomas et al.~\cite{Thomas-etal:2011},
as we were able to run our study on \postgres\footnote{
    We were not able to run this study on their other subject system,
    JHotDraw, as no official Git mirror of the repository is available.}.
We find similar topic distinctness scores for \postgres\ in our study,
even though we consider a later version of the source code.
This suggests that our approach is feasible,
as it captures distinct topics while not needing post-processing and is
always up-to-date with the source code repository.

\begin{comment}
% This is garbage
We investigated the problem in AspectJ further. One of the worst scoring
topics for distinctness was topic 90. The top 15 words from this topic were:
code, type, stack, method, names, string, index, array, exception, and constant.
Two of the highest ranked documents to this topic within $\theta$ are
changes to the {\small\texttt{org.aspectj.apache.bcel.classfile.Utility}} class.
The highest related change adds helper methods to the
{\small\texttt{org.aspectj.asm.internal.CharOperation}} class.
%Four out of the top five changes, however, were committed by "aclement".
\end{comment}

\begin{table}[h]
\renewcommand{\arraystretch}{1.3}
\footnotesize
\centering
\caption{Comparison of topic distinctness scores (RQ2)}
\begin{tabular}{lrr}
    \toprule
    System      & Release TD & Changeset TD \\
    \midrule
    \ant        & 2.31      & \textbf{3.17}      \\
    \aspectj    & \textbf{3.75}      & 2.78      \\
    \jodatime   & \textbf{1.34}      & 1.03      \\
    \postgres   & 2.59      & \textbf{3.56}      \\
    \bottomrule
\end{tabular}
\label{tab:tdscores}
\end{table}


\section{Threats To Validity}
\label{sec:threats}
% vim:syntax=tex

\begin{itemize}

\item Source code indexing
\item Topic distinctiveness as a measure of quality
\item LDA parameters chosen
\item Projects do not generalize

\end{itemize}


\section{Conclusion}
\label{sec:conclusion}
% vim:syntax=tex

In this paper conducted an exploratory study on modeling the topics of
changesets.
We used latent Dirichlet allocation (LDA) to extract linguistic
topics from changesets and source code.

We address two research questions regarding the topic modeling of changesets.
First, we investigated whether changeset copora were any different than
traditional release corpora, and what differences there might be.
For 2 of the 4 systems, we found that the changeset vocabulary was a superset
to the release vocabulary.
Next, we investigated whether a topic model trained on a changeset corpus
was more or less distinct than a topic model trained on a release corpus.
For 2 of the 4 systems, we found that the changeset corpus produced more
distinct topics, while for the other 2 it did not.

Future work includes expanding our evaluation and conducting an experiment
where we utilize these topic models, such as for bug localization.
Additional future work includes expanding our study to other systems, particularly ones that are not Java.
It seems unlikely that our results are specific to Java systems, though we cannot confirm this assumption without experimentation.
 


\section*{Acknowledgment}
% We thank the anonymous reviewers for their insightful comments and helpful suggestions.
This material is based upon work supported
by the U.S. Department of Education under Grant No.\ P200A100182 and
by the National Science Foundation under Grant No.\ 1156563.

%\IEEEtriggeratref{21}
\bibliographystyle{IEEEtran}
\bibliography{paper}

\end{document}
