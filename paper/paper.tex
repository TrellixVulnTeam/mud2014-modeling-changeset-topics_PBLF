\documentclass[conference]{IEEEtran}

\makeatletter
% IEEEtran.cls defines \labelindent for backward compatibility reasons
% Undefine \labelindent to allow the use of package enumitem
\let\labelindent\relax
\makeatother

\usepackage{balance}
\usepackage{subcaption}
\usepackage{booktabs}
\usepackage{cite}
\usepackage{color}
\usepackage{comment}
\usepackage{enumitem}
\usepackage[pdftex]{graphicx}
\usepackage{url}
\usepackage{listings}
\usepackage{array}

%\setlength{\parskip}{1pt}
%\setlength{\parsep}{1pt}

\definecolor{lightred}{RGB}{150,0,0}
\definecolor{lightgreen}{RGB}{0,150,0}
\definecolor{lightblue}{RGB}{0,0,150}

\lstdefinelanguage{diff}{
  morecomment=[f][\color{lightblue}]{@@},     % group identifier
  morecomment=[f][\color{lightred}]-,         % deleted lines
  morecomment=[f][\color{lightgreen}]+,       % added lines
  morecomment=[f][\color{black}]{---}, % Diff header lines (must appear after +,-)
  morecomment=[f][\color{black}]{+++},
}
\hyphenation{}

\newcommand{\attn}[1]{{\color{red}#1}}
\newcommand{\desc}[1]{{\emph{\color{blue}#1}}}
\newcommand{\needcite}{\attn{\tiny{[cite]}}}

\newcommand{\ant}{{Ant}}
\newcommand{\aspectj}{{AspectJ}}
\newcommand{\jodatime}{{Joda-Time}}
\newcommand{\postgres}{{PostgreSQL}}


\begin{document}
\title{Modeling Changeset Topics}
\author{
    \IEEEauthorblockN{
        Christopher S.\ Corley,
        Kelly L.\ Kashuda
    }
    \IEEEauthorblockA{
        Department of Computer Science\\
        The University of Alabama\\
        Tuscaloosa, AL 35487 USA\\
        \{cscorley, klkashuda\}@ua.edu
    }

    \and

    \IEEEauthorblockN{
        Daniel S.\ May
    }
    \IEEEauthorblockA{
        Department of Computer Science\\
        Swarthmore College\\
        Swarthmore, PA 19081 USA\\
        dmay1@swarthmore.edu
    }

    \and

    \IEEEauthorblockN{
        Nicholas A.\ Kraft
    }
    \IEEEauthorblockA{
        ABB Corporate Research\\
        Software Research Area\\
        Raleigh, NC 27606 USA\\
        nicholas.a.kraft@us.abb.com
    }
}


\maketitle

\begin{abstract}
Topic modeling has been applied to several areas of software engineering,
such as bug localization, feature location, triaging change requests,
and traceability link recovery.
Many of these approaches combine mining unstructured data, such as bug
reports, with topic modeling a snapshot (or release) of source code.
However, source code evolves, which causes models to become obsolete.
In this paper, we explore the approach of topic modeling \emph{changesets}
over the traditional release approach.
We conduct an exploratory study of four open source systems:
three written in Java and the last in C.
We investigate the differences in corpora in each project,
and evaluate the topic distinctness of the models.
\end{abstract}

\begin{IEEEkeywords}
Mining software repositories;
changesets;
topic modeling;
latent Dirichlet allocation
\end{IEEEkeywords}

\section{Introduction}
\label{sec:intro}
% vim:syntax=tex

Intro stuff \cite{Blei2003}

Since Latent Dirichlet Allocation (LDA) \cite{Blei2003} was first developed, it has been used on corpora from many disciplines to generate topic models [citations]. It has been used increasingly to address problems in the field of software engineering \cite{Binkley2014} from feature location \cite{Bassett2013} to author-topic modeling \cite{Steyvers2004}. However, topic models in software engineering are often developed by only using specific releases of source code. Every change that is made in the source code between releases could potentially cause previously-made topic models to become obsolete. This creates difficulties for those who want to use topic modeling in the software engineering because a model based on releases prevents up-to-date information leading up to a release from being utilized. In this paper, we explore developing topic models based upon the changes to the source code between commits (diffs) \cite{Thomas2011} rather than source code releases. This method allows the topic model to evolve dynamically over time without requiring the reprocessing of the entire corpus. While some research has been done on producing topic models based on the commit log comments \cite{Hindle2009}, our model instead focuses on the commit logs themselves and disregards the comments (I think this is right.).

The goal of this paper to determine if a topic model based on the change sets between commits can produce a topic model that is of equal quality to the common topic modeling based on analyzing releases. In this paper, we will compare results of topic models created through our model and the release model. The metric that we will use to determine quality is topic distinctness, which has been developed and used in topic modeling of source code \cite{Thomas2011}. Topic distinctness essentially measures how different one topic is from any other topic. We are aiming for the change set model to have at least equal topic distinctness as the standard release model because our model allows commits to be added to the already created topic model, resulting in far less computing time. Our ultimate goal is to develop a topic model that will allow authors to be linked to certain topics and features in the source code (info about the diagram Chris always has to draw on white board).

The corpora we analyzed was a large selection software that was freely collected online. The corpora we selected were based




\section{Background \& Related Work}
\label{sec:related}
% vim:syntax=tex

% this section totally not plagarised
%
In this section we describe the source code indexing process,
including key terms and individual steps.
% We particularly focus on the document extraction step.
We also provide an overview of Latent Dirichlet Allocation (LDA) as a text retrieval model
and review related work on its application in software engineering

\subsection{Source Code Indexing and Retrieval Process}

The source code indexing and modeling process,
as illustrated in Figure~\ref{fig:process},
defines a method for processing text (source code) into
a format suitable for efficient querying of the system, and
producing the result of such a query.
Terminology used in this section will first be defined, followed by an overview of each step of the process.

    \subsubsection{Terminology}
    We use the same terminology as Biggers et al.~\cite{Biggers-etal:2014}.
    They provide the following definitions:
    \begin{itemize}
    \item \textit{Term}: a sequence of letters and the basic unit of discrete data in a lexicon
    \item \textit{Token}: a sequence of non-whitespace characters; contains one or more terms
    \item \textit{Entity}: a source element such as a package, class, method, or file
    \item \textit{Identifier}: a token representing the name of an entity
    \item \textit{Comment}: a sequence of tokens delimited by language-
    specific markers, e.g., /* */
    \item \textit{String literal}: a sequence of tokens delimited by
    language-specific markers, e.g., “ ”
    \item \textit{Word}: the smallest free form in a language
    \end{itemize}
    Further, a term is described as being one of:
    word,
    abbreviation of a word,
    contraction of one or two words,
    or acronym of a series of words.

    \subsubsection{Indexing}
    The indexing process is illustrated on the left side of Figure~\ref{fig:process}.
    It converts source code into a corpus, or collection of documents.
    A document is a collection of terms that appear in a source code entity.
    Each entity in the source code will have an associated document in the resulting corpus.

    Typically, this indexing process has been used on source code files\needcite.
    \attn{Need stuff about Hindle and people using commit log messages} Another method that has been used to generate a corpus is one proposed by Hindle et al. ~\cite{Hindle2009} that creates tokens based on the commit log messages.
    In this method,
    commit log comments are extracted,
    word distributions are created based on those comments,
    then LDA is applied to create topic models.
    This model is similar to ours in that it uses tokens mined from commits rather than source code files.
    Nevertheless, we are unaware of any previous approach to indexing that is similar to what we propose,
    and our investigation of using indexing with LDA is the first of its kind in the software engineering domain.

    \subsubsection{Text Extraction}
    The text extraction stage receives as input source code and produces a list of tokens.
    The text extractor can be configured to tokenize some combination of
    the comments, string literals, and identifiers present in the source code.
    Biggers et al.~\cite{Biggers-etal:2014} evaluate the relative performance of an LDA-based FLT
    given text from different combinations of these sources
    and report that including text from all three sources generally provides the best performance.
    Hence, we choose not to configure a text extractor.
    Instead, we opt for using the entire document as a whole where applicable.

    \subsubsection{Preprocessing}
    The preprocessing stage takes each token produced by the text extractor
    and applies a series of processing steps, producing one or more terms.
    Common steps include~\cite{Marcus-etal:2004, Marcus-Menzies:2010}:
    \begin{itemize}
    \item \textit{Identifier Splitting}:
    divide a token into terms based on common coding conventions
    such as camel case and underscores
    and on punctuation or other symbols.
    The original term can be kept or discarded.
    \item \textit{Case Normalization}:
    convert all uppercase characters to lowercase, or vice-versa.
    \item \textit{Stop Word Filtering}:
    discard articles (e.g., ‘the’, or ‘a’),
    programming language specific keywords,
    and common words as defined by a stop-list.
    \item \textit{Word length filtering}:
    discard words that do not meet a lenght requirement,
    e.g. too short or two long.
    \item \textit{Stemming}:
    strip words of prefixes and suffixes to leave a common root.
    This allows different forms of the same word
    (e.g., ‘stemmer’, ‘stemming’)
    to be conflated to a single term
    (e.g., ‘stem’).
    \attn{Should probably mention Biggers-Kraft'10 here}
    \end{itemize}

    \subsubsection{Term Weighting}
    The term weighting step assigns a numeric value, or weight, to each item in the corpus.
    Common term weighting schemes described in the literature are
    binary, term count, term frequency, and
    term frequency-inverse document frequency (tf-idf)\needcite.

    \attn{Should probably talk about Bassett-Kraft'13 here}

    \subsubsection{Model Generation}
    Model generation is depicted in the center of Figure~\ref{fig:process}.
    This stage takes a corpus as input and processes it using a TR method
    to produce a model of the source code as output.
    The TR methods commonly used in this stage include the VSM\needcite,
    LSI~\cite{Deerwester-etal:1990},
    and LDA~\cite{Blei-etal:2003}.

    \subsubsection{Retrieval}
    The retrieval process is displayed in the right side of Figure 1.
    It takes a query as input and produces a ranked list of similar documents,
    each associated with a distinct source code entity, as output.

    \subsubsection{Querying}
    The query is a string generated manually or automatically (e.g., from an issue report).
    This query must be processed the same way as a document in the corpus,
    and so the query processor shown in Figure 1 will be internally similar to the document extractor.
    This produces a document associated with the query.
    It may have to be further processed to be directly comparable with elements in the TR model.

    \subsubsection{Ranking}
    The classifier shown in Figure~\ref{fig:process} is applied pairwise to
    the query document and each document in the model
    to score each pairing on similarity.
    These scores are then used to rank the documents by descending similarity.

\subsection{Latent Dirichlet Allocation}

Latent Dirichlet allocation (LDA)~\cite{Blei-etal:2003} is a generative topic model.
LDA models each document in a corpus of discrete data as a finite mixture over a set of topics
and models each topic as an infinite mixture over a set of topic probabilities.
That is, LDA models each document as a probability distribution
indicating the likelihood that it expresses each topic and
models each topic that it infers as a probability distribution
indicating the likelihood of a word from the corpus being assigned to the topic.

Inputs to LDA are:
\begin{itemize}
\item $D$, the documents
\item $K$, the number of topics
\item $\alpha$, the Dirichlet hyperparameter for topic proportions
\item $\beta$, the Dirichlet hyperparameter for topic multinomials
\end{itemize}

The documents provided are considered a bag-of-words,
represented as a vector of length $V\!$, the size of the vocabulary.
The original word ordering and grammatical information is discarded.
Outputs of LDA are:
\begin{itemize}
\item $\phi$, the term-topic probability distribution
\item $\theta$, the topic-document probability distribution
\end{itemize}

The parameters $\alpha$ and $\beta$ are used to control the smoothing of the model.
Topic distribution per document is influenced by $\alpha$,
and term distribution per topic is influenced by $\beta$.
Decreasing the value of $\alpha$ allows for fewer topics to be associated with a document,
while decreasing the value of $\beta$ generates topics that produce fewer terms
(increasing the number of topics needed to model the corpus).
Decreasing these values makes the computed probability distributions, $\phi$ and $\theta$,
more specific, increasing the decisiveness of the model.

\subsection{Topic Modeling Software}

Feature location as presented by Rajlich et al
is a way of locating concepts within code to increase understanding of the program as a whole~\cite{Rajlich-Wilde:2002}.
Linstead et al outlined a statistical model using LDA that was able to mine these concepts from source code~\cite{Linstead-etal:2007b}.
Linstead et al.~\cite{Linstead-etal:2007} used author-topic models to retrieve
developer contribtions from source code.
Lukins et al.~\cite{Lukins-etal:2008} implemented a way of using LDA to locate bugs in source
code that performed better than LSI-based information-retrieval
techniques.
Basset et al.~\cite{Bassett-Kraft:2013} extended this work
and studied various weightings of various terms in source code
to improve LDA-based feature location accuracy in five Java systems.
They found that a multiplier of eight for method names and a multiplier
of one for method calls is best for accuracy.

Hindle et al proposed that topic modeling ought be used on windowed topics~\cite{Hindle-etal:2009}, which were smaller in scope than a larger analysis based on a "global topic analysis" as much of the previous scholarship on topic models in SE used.
The model that Hindle et al used applied LDA to each separate commit's log comments independently then link the topics together in a way that is known as the Link model.
They found that there were certain topics that would not register in a larger analysis but would be important at certain times in the development process.

Thomas et al proposed that the model that ought be used for topic modeling should incorporate only the changes between versions~\cite{Thomas-etal:2011} to better represent the evolution of a corpus over time.
The diff model, as they named it, processes the source code so that all duplicated code is removed, then creates topic models based on the remaining altered source code.
This method created more distinct topics than the previously used models.
The metric that they used to measure the number of distinct topics was topic distinctness (More technical description of TD).
The higher topic distinctness a topic has the more distinct that topic is from other topics.
This is the measure that we will use to compare our model to a model based on releases.

\begin{comment}
Software source code is often compiled through a series of commits that change the source code between releases.
These commits can be of many sizes from a small change in a function to additions of many functions.
The person who commits these changes is seen as being the owner of the code~\cite{Corley-etal:2012}.
The owner of the code is assumed to have a greater working knowledge of the code in question than other members of the design team.
Being able to determine who would know most about a certain topic would allow someone working on a design team to know who is best fit to solve an issue pertaining to a certain topic.
Unfortunately, most of the topic modeling that is currently used to determine ownership is only based off of larger releases rather than intermediate changes.
These types of topic models can be problematic if you are working between releases.
Think for a second that you are a project manager at a software company with a rapidly approaching release deadline.
There is a problem of some sort with the code, and through the use of a topic model you could be able to identify precisely who should be delegated the task of fixing it.
With the type of dynamic topic model like we are proposing, this is easily attainable.
A model based on the change sets between commits would allow software developers never to have an obsolete model.
While there has been some research in the use of commit log comments~\cite{Hindle-etal:2009}, we are not aware of any analyses of topic models that have been created by using the differences of the source code between commits.
\end{comment}


\section{Case Study}
\label{sec:study}
% vim:syntax=tex

In this section we describe the design of a case study in which we
explore the relationship between ownership and linguistic topics in source code.
We describe the case study using the Goal-Question-Metric approach~\cite{Basili-etal:94}.
%The data for the case study is available in this paper's online
%appendix\footnote{\url{http://software.eng.ua.edu/data/ownership-topics}}.

\subsection{Definition and Context}

Our \textit{goal} is to explore the relationship between changeset topics and snapshot topics.
The \textit{quality focus} of the study is on informing development decisions and policy changes
that could lead to software with fewer defects.
The \textit{perspective} of the study is of a researcher, developer, or project manager who wishes
to gain understanding of the concepts or features implemented in the source code.
The \textit{context} of the study spans the version histories of 4 open source systems.

Toward achievement of our goal, we pose the following research questions:
\begin{enumerate}
    \item[]\hspace*{-20pt}\textit{RQ1: Do changeset- and snapshot-based corpora express the same terms?}
    \item[]\hspace*{-20pt}\textit{RQ2: Are topic models trained on changesets more distinct than topic models trained on a snapshot?}
\end{enumerate}
Basically, we want to know whether topic modeling changesets is as good as, or better than, topic modeling a snapshot.

In the remainder of this section we introduce the subjects of our study,
describe the setting of our study, and report our data collection and analysis procedures.

\subsection{Subject systems}

%%%%%%%%%%%%%%%%%%%%%%%%%%%%%%%%%%%%%%%%%%%%%%%%%%%%%%%%%%%%%%%%%%%%%%%%

\subsection{Study setup}

Rough outline of methodology

Gather corpora.
Get words from corpora.
Filter the words by splitting compound words and removing stop words.
Do not stem words because Biggers and Kraft found that the null stemmer was just as effective as six other stemmers in feature location techniques using LDA~\cite{Biggers-Kraft:2012}
Put filtered words from change sets into gensim to create topic models.
Put filtered words from release method into gensim to create topic models.
Test both types of models for topic distinctness.
Analyze effectiveness based on topic distinctness.


"Traditional metrics are, indeed, negatively correlated with the
measures of topic quality developed in this paper.  Our measures enable
new forms of model selection and suggest that practitioners developing
topic models should thus focus on evaluations that depend on real-world
task performance rather than optimizing likelihood-based measures." ~\cite{Chang-etal:2009}

\begin{figure}[ht]
\centering
\footnotesize
\begin{lstlisting}[language=diff, basicstyle=\ttfamily]
diff --git a/lao b/tzu
index 635ef2c..5af88a8 100644
--- a/lao
+++ b/tzu
@@ -1,7 +1,6 @@
-The Way that can be told of is not the eternal Way;
-The name that can be named is not the eternal name.
 The Nameless is the origin of Heaven and Earth;
-The Named is the mother of all things.
+The named is the mother of all things.
+
 Therefore let there always be non-being,
   so we may see their subtlety,
 And let there always be being,
@@ -9,3 +8,6 @@ And let there always be being,
 The two are the same,
 But after they are produced,
   they have different names.
+They both may be called deep and profound.
+Deeper and more profound,
+The door of all subtleties!
\end{lstlisting}
\caption{Example of a \texttt{git diff}. Black or blue lines denote metadata about the change, red lines (beginning with a single \texttt{-}) denote line removals, and green lines (beginning with a single \texttt{+}) denote line additions.}
\label{fig:diff}
\end{figure}

\begin{figure}[ht]
\em
\footnotesize
The Way that can be told of is not the eternal Way;
The name that can be named is not the eternal name.
The Named is the mother of all things.
The named is the mother of all things.
They both may be called deep and profound.
Deeper and more profound,
The door of all subtleties!
\caption{An example extracted changeset document before preprocessing}
\label{fig:diffdocument}
\end{figure}

\subsubsection{Text extraction from a release}

For our text extraction step on a release,
we simply use the entire contents of the document.
Unlike previous methods\needcite,
we do not parse the source code documents for classes, methods, and so on.
We do this because so-and-so-I-forget\needcite,
but also it leaves our technique language-independent.
Special characters such as braces and semicolons will be removed during
preprocessing.

\subsubsection{Text extraction from changesets}


To extract text from the changesets, we look at the output of viewing
the \texttt{git diff} between two commits.
Figure~\ref{fig:diff} shows an example of what a changeset might look
like in Git.

In our changeset text extractor, we only extract text from removed or added lines.
Context and metadata lines are ignored.
Figure~\ref{fig:diffdocument} shows what words would be extracted to make up the document without preprocessing
from the change shown in Figure~\ref{fig:diff}.
Note that we do not consider what type of document the text originates from,
only that it is text changed by the commit.

\subsubsection{Modeling}

For our topic modeling, we use the open source Python library Gensim~\cite{Gensim}.
Since Gensim's LDA implementation is based on the
Online LDA by Hoffman et al.~\cite{Hoffman-etal:2010},
it uses variational inference instead of a Collapsed Gibbs Sampler.
In order to ensure that the model converges for each document,
we allow LDA to see each document $10$ times by setting
Gensim's intialization parameter \texttt{passes} to this value.
We set the remaining LDA parameters as follows:
$100$ topics ($K$),
a symmetric $\alpha=0.01$,
$\beta$ is left as a default value of $1/K$ (also $0.01$).

%%%%%%%%%%%%%%%%%%%%%%%%%%%%%%%%%%%%%%%%%%%%%%%%%%%%%%%%%%%%%%%%%%%%%%%%

\subsection{Evaluation measures}

Following Thomas et al~\cite{Thomas-etal:2011}, we use topic distinctiveness
to evaluate our topic models.
Distinct topics are topics with dissimilar word vectors to all other topics in the model.
Thomas et al define topic distinctiveness (TD) of a topic $z_i$ as the mean
Kullback-Leibler (KL) divergence between the vectors $z_i$ and $z_j$, $\forall j \neq i$:

\begin{equation}
TD(\phi_{z_i}) = 
\frac{1}{K - 1}
\sum_{j=1,j \neq i}^{K}
KL(\phi_{z_i}, \phi_{z_j})
\label{eq:topicdistinctiveness}
\end{equation}



\section{Threats To Validity}
\label{sec:threats}
% vim:syntax=tex

Our study has limitations that impact the validity of our findings,
as well as our ability to generalize them.
We describe some of these limitations and their impacts.

Threats to construct validity concern the adequacy of the study procedure with regard to
measurement of the concepts of interest and can arise due to poor measurement design.
Threats to construct validity include the use of cosine similarity os our measure of similarity for corpora
and the use of topic distinctness to evaluate the topic models.

Threats to internal validity include possible defects in our tool chain and possible errors
in our execution of the study procedure,
the presence of which might affect the accuracy of our results and the conclusions we draw from them.
We controlled for these threats by testing our tool chain and by assessing the quality of our data.
Because we applied the same tool chain to all subject systems, any errors are systematic and are unlikely
to affect our results substantially.

Another threat to internal validity pertains to the value of $K$ that we selected for all models trained.
We decided that the changeset and release models should have the same $K$
to help facilitate evaluation and give each model equal footing.

Threats to external validity concern the extent to which we can generalize our results.
The subjects of our study comprise four open source systems in two langauges,
so we cannot generalize our results to systems implemented in other languages.
However, the systems are of different sizes, are from different domains, and
have characteristics in common with those of systems developed in industry.



\section{Conclusion}
\label{sec:conclusion}
% vim:syntax=tex

In this paper conducted an exploratory study on modeling the topics of
changesets.
We used latent Dirichlet allocation (LDA) to extract linguistic
topics from changesets and snapshots (releases).

We addressed two research questions regarding the topic modeling of changesets.
First, we investigated whether changeset copora were any different than
traditional release corpora, and what differences there might be.
For two of the systems, we found that the changeset vocabulary was a superset
to the release vocabulary.
We measured the cosine distance of each distribution of words,
and found for 3 of the systems low (between 0.003 to 0.07),
while the last was much higher than the others (over 0.33).
Next, we investigated whether a topic model trained on a changeset corpus
was more or less distinct than a topic model trained on a release corpus.
For 2 of the 4 systems, we found that the changeset corpus produced more
distinct topics, while for the other 2 it did not.


Future work includes expanding our evaluation and conducting an experiment
where we utilize these topic models, such as for bug localization.
Additional future work includes expanding our study to other systems, particularly ones that are not Java.
It seems unlikely that our results are specific to Java systems, though we cannot confirm this assumption without experimentation.
This expansion should also include an investigation into why some
changeset topic models are more distinct than others.
 


\section*{Acknowledgment}
% We thank the anonymous reviewers for their insightful comments and helpful suggestions.
This material is based upon work supported
by the U.S. Department of Education under Grant No.\ P200A100182 and
by the National Science Foundation under Grant No.\ 1156563.

%\IEEEtriggeratref{21}
\bibliographystyle{IEEEtran}
\bibliography{paper}

\end{document}
