% vim:syntax=tex
Software developers commonly find themselves dealing with large repositories that date back extensively, before delving into a project or specific task (e.g. feature location, maintenance, etc.)
Navigating through these repositories can be a time consuming task since their organization, or lack there of, can be hard to understand.
Topic models have been used to address this problem by identifying the underlying topics in the repositories.
These underlying topics are found using a statistic analysis, through the use of algorithms that are designed to uncover a thematic structure.
Finding this thematic structure helps developers to navigate and better understand the repositories they are working with.
Topics are discovered by taking a distribution over the vocabulary, which creates a set of topics for each document in the corpus.
With each document representing multiple topics, connections are made making the documents easier to navigate.
While topic models have addressed these problems (e.g., feature location program comprehension, maintenance, etc.) models are typically built on a snapshot of code, or a release.
A problem with this is, source code is not static, and the drastic amount of changes that occur in between releases forces the models to be rebuilt from scratch.
Given the nature of these problems, we propose that we use the individual change-sets to continually update the source code.
A change-set can be thought of as indexing into the documents where changes have occurred, and transforming the indexes from version A to version B.
By doing this, we gain a topic model that evolves dynamically over time without requiring the reprocessing of the entire corpus.

The goal of this paper to determine if a topic model based on the change-sets between commits can produce a topic model that is of equal quality to the common topic modeling based on analyzing releases.
In this paper, we will compare results of a standard release model and our change-set model.
The metric that we will use to determine quality is topic distinctness,
which has been developed and used in topic modeling of source
code~\cite{Wei-etal:2010, Thomas-etal:2011, Chuang-etal:2012}.
Topic distinctness measures how distinct a newly discovered topic is in
comparison to the other, already discovered topics.
We use this metric as a  tool to compare our change-set model to a model built upon a release.



% let's not elude to this yet
%Our ultimate goal is to develop a topic model that will allow authors to
%be linked to certain topics and features in the source code (info about
%the diagram Chris always has to draw on white board).


\attn{Research Q1:
    Are the most influential words of file-based corpora
    interacted with more over time in changeset-based corpora?
    }
\attn{Research Q2:
    Are topics drawn from changeset-based corpora more distinct than
    topics drawn from file-based corpora?
    }
