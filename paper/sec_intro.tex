% vim:syntax=tex
Software developers are often confronted with maintenance tasks that involve navigation of repositories that preserve vast amounts of project history.
 Navigating these software repositories can be a time-consuming task, because their organization (or lack thereof) can be difficult to understand.
 Fortunately, topic models such as latent Dirichlet allocation (LDA)~\cite{Blei-etal:2003} can help developers to navigate and understand software repositories by discovering topics (word distributions) that reveal the thematic structure of the data~\cite{Linstead-etal:2007,Thomas-etal:2011,Hindle_etal:2012}.


When modeling a source code repository, the corpus typically represents a snapshot of the code.
 That is, a topic model is often trained on a corpus that contains documents that represent files from a particular version of the software.
 Keeping such a model up-to-date is expensive, because the frequency and scope of source code changes necessitate retraining the model on the updated corpus.
 However, it may be possible to automate certain maintenance tasks without a model of the complete source code.
For example, when assigning a developer to a change task, a topic model can be used to associate developers with topics that characterize their previous changes.
In this scenario, a model of the changesets owned by each developer may be more useful than a model of the files changed by each developer.
 Moreover, as a typical changeset is smaller than a typical file, a changeset-based model is less expensive to keep current than a file-based model.



Toward the goal of automating software maintenance tasks using changeset-based models, in this paper we qualitatively compare topic models trained on corpora of changesets to those trained on files.
 For our comparison we consider vocabulary measures, which indicate whether term distributions in the changeset corpora match those in the file corpora, and topic distinctness~\cite{Wei-etal:2010,Thomas-etal:2011,Chuang-etal:2012}, which measures how distinct one topic in a model is from another.
 Models with higher topic distinctness values are desirable, because distinct topics are more useful in differentiating among the documents in a corpus than are similar topics.

