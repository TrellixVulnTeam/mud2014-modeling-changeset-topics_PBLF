% vim:syntax=tex
In this section, we review the current state of source code topic modeling scholarship to better explain the exigency of our work.

Software source code is often compiled through a series of commits that change the source code between releases.
These commits can be of many sizes from a small change in a function to additions of many functions.
The person who commits these changes is seen as being the owner of the code \cite{Corley2012}.
The owner of the code is assumed to have a greater working knowledge of the code in question than other members of the design team.
Being able to determine who would know most about a certain topic would allow someone working on a design team to know who is best fit to solve an issue pertaining to a certain topic.
Unfortunately, most of the topic modeling that is currently used to determine ownership is only based off of larger releases rather than intermediate changes.
These types of topic models can be problematic if you are working between releases.
Think for a second that you are a project manager at a software company with a rapidly approaching release deadline.
There is a problem of some sort with the code, and through the use of a topic model you could be able to identify precisely who should be delegated the task of fixing it.
With the type of dynamic topic model like we are proposing, this is easily attainable.
A model based on the change sets between commits would allow software developers never to have an obsolete model.
While there has been some research in the use of commit log comments \cite{Hindle2009}, we are not aware of any analyses of topic models that have been created by using the differences of the source code between commits.


\subsection{ LDA and Software Engineering }

Latent Dirichlet allocation (LDA) \cite{Blei2003} is a generative probabilistic model that generates different topics and subtopics for a corpus.
LDA can allow someone to quickly view which are the most important concepts in a collection of words.
Shortly after Blei et al introduced LDA, the Author-Topic model developed to match topics and the author of those topics \cite{Rosen-Zvi2004} \cite{Steyvers2004}.
The author-topic model introduced has many applications in the realm of software engineering and was applied to developer contribution by Linstead et al
\cite{Linstead2007}.
In this paper, Linstead et al showed that Author-Topic models could successfully be used to match developers to specific topics in the Eclipse 3.
0 source code.
This idea can be extended to ownership of the code \cite{Corley2012}, which is an important distinction to be made when dealing with commits as we will be in this paper.


\subsection{ Modeling topic evolution }

Hindle et al proposed that topic modeling ought be used on windowed topics \cite{Hindle2009}, which were smaller in scope than a larger analysis based on a "global topic analysis" as much of the previous scholarship on topic models in SE used.
The model that Hindle et al used applied LDA to each separate commit's log comments independently then link the topics together in a way that is known as the Link model.
They found that there were certain topics that would not register in a larger analysis but would be important at certain times in the development process.

Thomas et al proposed that the model that ought be used for topic modeling should incorporate only the changes between versions \cite{Thomas2011} to better represent the evolution of a corpus over time.
The diff model, as they named it, processes the source code so that all duplicated code is removed, then creates topic models based on the remaining altered source code.
This method created more distinct topics than the previously used models.
The metric that they used to measure the number of distinct topics was topic distinctness (More technical description of TD).
The higher topic distinctness a topic has the more distinct that topic is from other topics.
This is the measure that we will use to compare our model to a model based on releases.

\subsection{Feature location}

Feature location as presented by Rajlich et al
is a way of locating concepts within code to increase understanding of the program as a whole \cite{Rajlich2002}.
Linstead et al outlined a statistical model using LDA that was able to mine these concepts from source code \cite{Linstead2007_2}.
Lukins et al implemented a way of using LDA to locate bugs in source code that performed better than LSI-based information-retrieval techniques \cite{Lukins2008}.
Basset et al extended this work and studied various weightings of various terms in source code to improve LDA-based feature location accuracy in five Java systems \cite{Bassett2013}.
They found that a multiplier of eight for method names and a multiplier of one for method calls is best for accuracy.



