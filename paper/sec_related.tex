% vim:syntax=tex

In this section we provide an overview of latent Dirichlet allocation and review closely related work.

\subsection{Latent Dirichlet Allocation}

Latent Dirichlet allocation (LDA)~\cite{Blei-etal:2003} is a generative topic model.
LDA models each document in a corpus of discrete data as a finite mixture over a set of topics
and models each topic as an infinite mixture over a set of topic probabilities.
That is, LDA models each document as a probability distribution
indicating the likelihood that it expresses each topic and
models each topic that it infers as a probability distribution
indicating the likelihood of a word from the corpus being assigned to the topic.

%Inputs to LDA include a corpus and $K$, the number of topics. LDA represents each document in the corpus as a bag-of-word (multiset) and thus disregards word order and structure. Outputs of LDA include $\phi$, the term-topic probability distribution, and $\theta$, the topic-document probability distribution.


\subsection{Topic Models in Software Maintenance}

%Feature location as presented by Rajlich et al.\ is a way of locating concepts within code to increase understanding of the program as a whole~\cite{Rajlich-Wilde:2002}.
%Linstead et al.\ outlined a statistical model using LDA that was able to mine these concepts from source code~\cite{Linstead-etal:2007b}.
%Linstead et al.~\cite{Linstead-etal:2007} used author-topic models to retrieve
%developer contribtions from source code.
%Lukins et al.~\cite{Lukins-etal:2008} implemented a way of using LDA to locate bugs in source
%code that performed better than LSI-based information-retrieval
%techniques.
%Basset et al.~\cite{Bassett-Kraft:2013} extended this work
%and studied various weightings of various terms in source code
%to improve LDA-based feature location accuracy in five Java systems.

Thomas et al.~\cite{Thomas-etal:2011} describe an approach to modeling the evolution of source code topics using LDA. Their \textit{Diff} model outperforms the Hall topic evolution model~\cite{Hall_etal:2008} in the context of software repositories, because the \textit{Diff} model trains topic models on the changesets between two snapshots, rather than on individual snapshots. That is, for a particular source code file, \textit{Diff} trains a topic model on a document that represent the changes between consecutive versions of the file. Consequently, the \textit{Diff} model eliminates the issues related to data duplication that arise in the Hall model, which trains a topic model on all versions of the (whole) file. Thomas et al.\ demonstrate the efficacy of the \textit{Diff} model via a comparative study with the Hall model. Their evaluation measures include topic distinctness, which we define in Section~\ref{sec:study}.

Hindle et al.~\cite{Hindle_etal:2012} validate the use LDA topics during software maintenance via a study at Microsoft. Their focus is on stakeholder validation of topics --- i.e., they seek confirmation that LDA topics are interpretable by stakeholders (e.g., developers or managers) and relevant to the requirements implemented by the modeled source code. Previous work by Hindle et al.~\cite{Hindle-etal:2009} describes an approach to modeling the evolution of software topics using commit messages rather than source code.

\begin{table*}[ht]
\renewcommand{\arraystretch}{1.3}
\footnotesize
\centering
\caption{Subject systems version and corpora description}
\begin{tabular}{lll rr rr rr}
    \toprule
            & Snapshot & Commit & Snapshot No.  & Changeset No. & Snapshot No.    & Changeset No.  & Snapshot No.    & Changeset No.\\
     System    & Version & SHA  & Documents   & Documents    & Unique Terms  & Unique Terms  & Total Terms   & Total Terms \\
    \midrule
    \ant        & 1.9.4   & 1c927b15 & 2208      & 12,996     & 17,986         & 74,681         & 1,066,446       & 11,801,353 \\
    \aspectj    & 1.8.0   & 5a5bef1e & 10130     & 7,650      & 22,855         & 25,071         & 4,825,289       & 10,583,008 \\
    \jodatime   & 2.3     & b0fcbb95 & 402       & 1,750      & 9,298          & 11,385         & 493,131        & 5,541,330 \\
    \postgres  & 9.3.4   & d4f8dde3 & 4080      & 36,870     & 84,591         & 164,703        & 6,644,409       & 59,850,328 \\
    \bottomrule
\end{tabular}
\label{tab:systems}
\end{table*}

Although our work is preliminary, we believe that it is the first to consider modeling changesets in lieu of snapshots to support software maintenance. Like Rao et al.~\cite{Rao-etal:2011}, we are targeting problems that require an up-to-date topic model. Thus, the expense of training a topic model is a key consideration for us, unlike for Thomas et al.~\cite{Thomas-etal:2011} or Hindle et al.~\cite{Hindle-etal:2009,Hindle_etal:2012}.
