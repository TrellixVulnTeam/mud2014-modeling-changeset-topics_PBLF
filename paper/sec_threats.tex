% vim:syntax=tex

Our study has limitations that impact the validity of our findings,
as well as our ability to generalize them.
We describe some of these limitations and their impacts.

\begin{itemize}

\item Source code indexing
\item Topic distinctiveness as a measure of quality
\item LDA parameters chosen
\item Projects do not generalize

\end{itemize}

Threats to construct validity concern the adequacy of the study procedure with regard to
measurement of the concepts of interest and can arise due to poor measurement design.
Threats to construct validity include the use of cosine similarity os our measure of similarity for corpora
and the use of topic distinctiveness to evaluate the topic models.

Threats to internal validity include possible defects in our tool chain and possible errors
in our execution of the study procedure,
the presence of which might affect the accuracy of our results and the conclusions we draw from them.
We controlled for these threats by testing our tool chain and by assessing the quality of our data.
Because we applied the same tool chain to all subject systems, any errors are systematic and are unlikely
to affect our results substantially.

Another threat to internal validity pertains to the value of $K$ that we selected for all models trained.
We decided that the changeset and release models should have the same $K$
to help facilitate evaluation and give each model equal footing.

Threats to external validity concern the extent to which we can generalize our results.
The subjects of our study comprise four open source systems in two langauges,
so we cannot generalize our results to systems implemented in other languages.
However, the systems are of different sizes, are from different domains, and
have characteristics in common with those of systems developed in industry.

